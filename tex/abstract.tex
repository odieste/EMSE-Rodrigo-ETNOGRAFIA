\begin{abstract}
\textbf{\textendash\textendash \textit{Antecedentes: }}Han pasado varias d�cadas desde que se llev� a cabo el primer experimento en Ingenier�a de Software (SE). Desde entonces, ha habido un importante repunte en el n�mero de experimentos realizados en la academia y �ltimamente en la industria. No obstante, algunos investigadores en SE consideran que a�n existe incertidumbre respecto a las actividades a realizar antes, durante y despu�s de un experimento. \textbf{\textit{Objetivo: }}Con el prop�sito identificar y validar la problem�tica en torno a la Experimentaci�n en Ingenier�a de Software (ESE), hemos llevado a cabo varios estudios emp�ricos exploratorios, motivados en los indicios de formalidad identificados en el proceso de experimentaci�n de las disciplinas tradicionales. \textbf{\textit{Metodolog�a: }}Se realiz� un estudio exploratorio basado en m�todos emp�ricos mixtos para: (1) indagar sobre la problem�tica en torno a la ESE dentro de un grupo de investigaci�n representativo en la comunidad de SE emp�rica (case study 1), (2) aprender del proceso de experimentaci�n en una disciplina tradicional e identificar la problem�tica particular de la Experimentaci�n en SE (case study 2), y (3) validar la generalidad de la problem�tica identificada en la ESE (survey). \textbf{\textit{Resultados: }}Se obtuvieron modelos conceptuales y de proceso que permitieron entender de forma expl�cita la experimentaci�n llevada a cabo en la SE, as� como sus carencias respecto a una disciplina experimental tradicional y finalmente se comprob� que las carencias identificadas son generales. \textbf{\textit{Conclusiones: }} Existen actividades y conceptos utilizados en el proceso de experimentaci�n de las disciplinas experimentales tradicionales que son semejantes a los utilizados en el proceso de experimentaci�n es SE. Sin embargo, existen actividades y conceptos podr�an ser aplicados en la SE para mejorar su proceso de experimentaci�n.
\end{abstract}