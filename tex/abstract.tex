\begin{abstract}
\textbf{\textendash\textendash \textit{Background: }}Experimentation is a well-established research method in Software Engineering (SE) nowadays. In fact, researchers in SE have started to perform an introspection of its discipline, as it happened in other traditional disciplines. However, there remains important details within experimentation that apparently have not yet been fully suitable in SE and still represent a challenge. \textbf{\textit{Aim: }}. To observe specific details about the experimental protocol applied in SE and other tradicional discipline, and extrapolating findings to the entire SE community. \textbf{\textit{Method: }}An exploratory study based on empirical mix methods was conducted, which involved the identification of activities in practice regarding the experimentation in SE and other traditional discipline, and the validation of findings within empirical software engineering community. \textbf{\textit{Results: }}Conceptual and process models were obtained and made it possible to identify some constraints within the SE experimental process, which was validated by the ESE community. \textbf{\textit{Conclusions: }}Formalization of the SE experimentation process is possible achieving harmonization of its experimental tasks through specific protocols for each research area.
\end{abstract}
\rodrinote{PENDIENTE DE REVISAR EL RESUMEN}