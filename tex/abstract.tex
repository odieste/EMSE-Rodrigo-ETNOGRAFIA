\begin{abstract}
\textbf{\textendash\textendash \textit{Background: }}Several decades have passed since the first experiment was carried out in software engineering (SE). Since then and up until now, there is a significant upturn regarding the number of experiments undertaken in academy and lately in industry. However, some SE researchers consider that there is still uncertainty about what activities have to be performed before, during and after an experiment. \textbf{\textit{Aim: }}Identify and validate constraints surrounding the experimentation in SE, taking the signs of formality identified into the experimental process of tradicional disciplines as reference. \textbf{\textit{Method: }}An exploratory study based on empirical mix methods was conducted, which involved the identification of the activities' complexity in practice regarding the experimentation in SE; learning about the experimental process performed in a traditional discipline to identify particular constraints within experimentation in SE through comparison; and the validation of experimental SE constraints within empirical software engineering (ESE) community. \textbf{\textit{Results: }}Conceptual and process models obtained around the observation and the participation of the SE experimental activities and other traditional discipline, respectively, made it possible to identify some constraints within the SE experimental process through comparison. Additionally, the ESE community validated the presence of such constraints. \textbf{\textit{Conclusions: }}There are activities and concepts used in the experimental process of traditional disciplines that could be applied in the experimental SE in order to improve its process.
\end{abstract}
%\textbf{\textendash\textendash \textit{Antecedentes: }}Han pasado varias d�cadas desde que se llev� a cabo el primer experimento en Ingenier�a de Software (SE). Desde entonces, ha habido un importante repunte en el n�mero de experimentos realizados en la academia y �ltimamente en la industria. No obstante, algunos investigadores en SE consideran que a�n existe incertidumbre respecto a las actividades a realizar antes, durante y despu�s de un experimento. \textbf{\textit{Objetivo: }}Identificar y validar la problem�tica en torno a la Experimentaci�n en SE, tomando como referencia los indicios de formalidad identificados en el proceso de experimentaci�n de las disciplinas tradicionales. \textbf{\textit{Metodolog�a: }}Se realiz� un estudio exploratorio basado en m�todos emp�ricos mixtos para: (1) indagar sobre la problem�tica en torno a la experimentaci�n en SE dentro de un grupo de investigaci�n representativo en la comunidad de SE emp�rica (ESE), (2) aprender del proceso de experimentaci�n en una disciplina tradicional e identificar la problem�tica particular de la Experimentaci�n en SE, y (3) validar la generalidad de la problem�tica identificada en la experimentaci�n en SE. \textbf{\textit{Resultados: }}Se obtuvieron modelos conceptuales y de proceso que permitieron entender de forma expl�cita la experimentaci�n llevada a cabo en la SE, as� como sus carencias respecto a una disciplina experimental tradicional y finalmente se comprob� que las carencias identificadas en la SE son generales. \textbf{\textit{Conclusiones: }} Existen actividades y conceptos utilizados en el proceso de experimentaci�n de las disciplinas tradicionales que son semejantes a los utilizados en el proceso de experimentaci�n es SE. Sin embargo, existen actividades y conceptos que podr�an ser aplicados en la SE para mejorar su proceso de experimentaci�n.