\begin{abstract}
\textbf{\textendash\textendash \textit{Background: }}Experimentation is a well-established research method in Software Engineering (SE) nowadays. As a result, researchers in SE have started to perform introspection of its discipline, as it happened in other traditional disciplines. However, essential details remain within experimentation that have not been entirely suitable in SE and still represent a challenge. \textbf{\textit{Aim: }}. To observe specific details about the experimental protocol applied in SE and another traditional discipline and extrapolate findings to the entire SE community. \textbf{\textit{Method: }}An exploratory study based on mixed empirical methods was conducted, which involved identifying activities in practice regarding the experimentation in SE and another traditional discipline and the validation of findings within the empirical software engineering community. \textbf{\textit{Results: }}Conceptual and process models were obtained and made it possible to identify some constraints within the SE experimental process, validated by the ESE community. \textbf{\textit{Conclusions: }}Formalizing the SE experimentation process is possible, achieving harmonization of its experimental tasks through specific protocols for each research area.
\end{abstract}
