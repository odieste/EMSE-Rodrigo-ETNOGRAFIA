\begin{abstract}
\textbf{\textendash\textendash \textit{Antecedentes: }}Han pasado varias d�cadas desde que se llev� a cabo el primer experimento en Ingenier�a de Software (SE). Desde entonces, el n�mero de experimentos ha tenido un importante repunte, principalmente en la academia y �ltimamente en la industria. No obstante, algunos investigadores consideran que la Experimentaci�n en SE ha presentado dificultades desde un inicio, principalmente en la incertidumbre respecto a las actividades a realizar antes, durante y despu�s de un experimento. \textbf{\textit{Objetivo: }}Los indicios de formalidad identificados en la experimentaci�n de las disciplinas tradicionales nos ha motivado a llevar a cabo estudios emp�ricos exploratorios para identificar y validar la problem�tica en torno al proceso de experimentaci�n en SE. \textbf{\textit{Metodolog�a: }}Se realizaron tres estudios emp�ricos exploratorios para: (1) indagar sobre la problem�tica en torno a la experimentaci�n en SE (case study 1), (2) indagar sobre el proceso de experimentaci�n en una disciplina tradicional (case study 2) e identificar la problem�tica particular de la SE , y (3) comprobar la generalidad de la problem�tica identificada en la experimentaci�n en SE (survey). \textbf{\textit{Resultados: }}Se obtuvieron modelos conceptuales y de proceso que describen la experimentaci�n llevada a cabo en la SE y en una disciplina experimental tradicional. \textbf{\textit{Conclusiones: }} Existen actividades y conceptos utilizados en el proceso de experimentaci�n en una disciplina experimental tradicional que son semejantes a los utilizados en el proceso de experimentaci�n es SE. Sin embargo, existen actividades y conceptos podr�an ser aplicados en la SE para mejorar su proceso de experimentaci�n, particularmente en la formalidad de la distribuci�n de las actividades entre los roles que participan en el proceso.
\end{abstract}