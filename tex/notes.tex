\section{Actions and Findings}
\textbf{Acci�n = Revisi�n de literatura}\\

\textbf{Hallazgos:}\\
\begin{itemize}
	\item \textit{Diversidad terminol�gica con que las distintas fuentes se refer�an a actividades similares del proceso experimental.}
	\item \textit{carencia de textos sobre experimentaci�n en SE.}
	\item 
\end{itemize}


\textbf{Acci�n = Revisi�n del material experimental}\\

\textbf{Hallazgos:}\\
\begin{itemize}
	\item \textit{No existe una pol�tica o herramienta para la gesti�n del material.}
	\item \textit{Las actividades que generaron el material experimental han ido evolucionando y mejorando con el tiempo.}
	\item \textit{La extensi�n (entre 100 a 120 p�ginas) y complejidad del planteamiento de los experimentos.}
	\item \textit{Descripci�n general de las actividades a ser realizadas en el experimento.}
	\item \textit{El material experimental no da una gu�a sobre el orden de ejecuci�n de las actividades experimentales.}
	\item \textit{La diversidad terminol�gica encontrada causa incertidumbre sobre la fiabilidad de las fuentes revisadas.}
	\item \textit{El material experimental no presentaba una gu�a clara de c�mo hacer un experimento.}
	\item \textit{Existe una diversidad de aproximaciones y enfoques sobre la realizaci�n de experimentos.}
\end{itemize}

\textbf{Acci�n = Revisi�n de material experimental espec�fico}\\

\textbf{Hallazgos:}\\
\begin{itemize}
	\item \textit{Realizar en la pr�ctica una replicaci�n experimental sin duda permiti� entender de forma m�s did�ctica su proceso completo y las actividades inherentes.}
	\item \textit{aclarar la concepci�n de la experimentaci�n obtenida en la revisi�n de la literatura referente y del material experimental del GrISE.}
	\item \textit{comprobar la dificultad que representa llevar a cabo un experimento en la pr�ctica, m�s a�n al tratar de recrear fielmente un experimento ya realizado en un entorno diferente.}
	\item \textit{no existe una gu�a clara y auto sustentable que permita realizar una replicaci�n literal externa, siendo necesaria la realizaci�n de una replicaci�n conjunta.}
	\item \textit{La ejecuci�n de una replicaci�n en la pr�ctica permiti� observar la generalidad de los textos, la falta de detalle del material experimental y la presencia de un conocimiento t�cito particular de cada investigador}
\end{itemize}

\textbf{Acci�n = Entrevistas personales con los miembros del GrISE}\\

\textbf{Hallazgos:}\\
\begin{itemize}
	\item \textit{los experimentadores de mayor experiencia tienen un conocimiento m�s amplio y detallado sobre las actividades que llevan a cabo en la experimentaci�n en SE.}
	\item \textit{el detalle de actividades obtenido en el flujograma del proceso experimental de la Figura \ref{fig-proceso-exp}, indica que los textos sobre experimentaci�n en SE existentes est�n lejos de mostrar el trabajo real que implica la experimentaci�n en la pr�ctica de principio a fin.}
\end{itemize}