\section{Introduction}\label{sec-introduction}
SE researchers have started to perform introspection on how empirical Software Engineering (SE) functions, as it had happened in other traditional disciplines, such as Medicine \cite{Altman-1998-statistical}. As Kitchenham et al. \cite{Kitchenham2002-GuideLinesESE} point it out: \textquotedblleft If researchers have difficulty in a discipline such as medicine, which has a rich history of empirical research, it is hardly surprising that software engineering researchers have problems\textquotedblright.

Los pocos estudios existentes han abordado principalmente aspectos estad�sticos. Por ejemplo, Vegas et al. \cite{Vegas-2016-Crossover-Designs-ESE} han estudiado en qu� medida los dise�os cross-over son correctamente analizados. Del mismo modo, Dyb{\aa} et al. \cite{Dyba2006} han analizado la dificultad de sacar conclusiones v�lidas a partir de experimentos realizados en SE con bajo poder estad�stico. Desde una perspectiva m�s general, Reyes et al. \cite{Reyes-2018-Statistical-Errors-in-SE} han determinado emp�ricamente la existencia de distintos tipos de errores en experimentos de SE. J{\o}rgensen et al. \cite{Jorgensen-2016-Incorrects-Results-SEE} llegan a afirmar que una gran parte de los resultados experimentales en SE son falsos.

 This issue is not privative of SE. Other experimental disciplines, such as medicine \cite{Welch-1996-review}, psychology \cite{Bakker-2011-mis}, and psychiatry \cite{Ercan-2008-misusage}, reported similar problems at some point.
 
 Sin embargo, los aspectos matem�tico-estad�sticos solo representan parte de la complejidad de la investigaci�n experimental, y hasta cierto punto lo que m�s f�cilmente puede ser gestionado. To the best of our knowledge, no existe ning�n estudio en SE que aborde el proceso experimental en general, esto es, el modo en que los experimentadores en SE conciben, planifican, ejercitan y documentan una investigaci�n experimental. Lo m�s pr�ximo a esta investigaci�n es la abundante bibliograf�a acerca de replicaciones experimentales, donde tambi�n se se�alan varias carencias, como por ejemplo: dise�os y reportes experimentales inadecuados para una replicaci�n \cite{Miller-2005-replicating-SE-experiments}, diversidad terminol�gica en los experimentos que imposibilitan su replicaci�n \cite{Demagalhaes-2015-SMS-Replications}. De nuevo, esto es com�n en otras disciplinas.
 
 Este art�culo tiene como objetivo averiguar el modo en que los investigadores de SE afrontan la investigaci�n experimental, y cu�les son las similitudes y diferencias entre dicha pr�ctica y la pr�ctica de otras disciplinas asentadas.
 
 Para ello, we performed a mix-method study. First, we carried out a comparative observational study between the experimentation processes in SE and a traditional experimental discipline (Biotechnology). Second, we surveyed the experimental software community in order to validate and generalize the findings obtained in the first study.
 
 The results of this study suggest that the formalization of the experimental process in SE is possible in specific areas, where the experimental tasks are harmonically orchestrated given the particular protocols used in there.

This study contributes to SE research through the identification of experiment practices in another traditional discipline, which could enhance the formalization of the experiment process in improvement comes out of \rodrinote{DEJAR EN BLANCO}

The remainder of the paper is organized as follows: Section \ref{sec-background} introduces the context which motivated the research. Research methodology is described in Section \ref{sec-methology}. Section \ref{sec-ESE-etnography} shows the results of the ethnographic study carried out in a representative SE research group. Afterward, Section \ref{sec-survey} details the results of the validation about SE experimentation issues identified. Section \ref{sec-bio-etnography} specifies the ethnographic study performed in specific Biotechnology areas. A summarized discussion about threats to validity identified is presented in Section \ref{sec-threats}. Finally, the discussion and main conclusions are presented in Section \ref{sec-discussion-conclusions}.
\rodrinote{OSCAR POR FAVOR REVISA NUEVAMENTE LA INTRODUCCI�N}.