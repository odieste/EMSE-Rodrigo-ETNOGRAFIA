\section{Introduction}\label{section-introduction}
More than 50 years have passed since the first experiment was carried out into SE \cite[p.~7]{Wohlin2012-Experimentation} as mechanism to assess current methods and techniques of software development in order to improve software product quality. Several experiments have been performed both in academy \cite{Sjoberg2005-surveyexperimentsESE}, as in industry \cite{dieste-2013-industry-experiments}. Nowadays it is difficult to find conferences or journals that does not regularly include experimental studies. This situation prompted that SE researchers, interested in experimentation, had to acquire adequate training to take advantage of the different sources of information regarding the experimental process (e.g.: booklets, textbooks, guidelines, papers, among others.)\cite[p.~3-4]{Kitchenham-2015-Literature-Review-Book}. Therefore, the experimentation in SE is apparently a seated methodology.

However, the current situation around the Experimentation in SE is far less effective than it could or should be \cite{Kitchenham2002-GuideLinesESE}. As Kitchenham et al. \cite{Kitchenham2002-GuideLinesESE} said: \textquotedblleft If researchers have difficulty in a discipline such as medicine, which has a rich history of empirical research, it is hardly surprising that software engineering researchers have problems\textquotedblright. We all know that empirical studies en SE, such as experimentation, can cover a wide range of execution possibilities. However, in addition to this characteristics, they include the use of specialized vocabulary, and the decision making conflicting (on many occasions) with similar empirical studies, during the stages of design implementation and analysis; which causes that each process is completely different .

Por todos es bien sabido que los estudios emp�ricos en SE, como la experimentaci�n, son diversos en car�cter como ocurre en otras disciplinas. However, son utilizadas terminolog�as peculiares, se toman decisiones durante las etapas de dise�o ejecuci�n y an�lisis, en muchas ocasiones contrapuestas a las de otros estudios emp�ricos del mismo tipo, lo que causa que su proceso sea completamente distinto [REF KEY NOTE Wohlin].

More specifically, there are uncertainties about how to perform experiments in terms of design \cite{mian-200-computarized-infraestructure} and running \cite{Mendonca-2008-FIRE}. According some studies the experimentation in SE has several deficiencies, for example: the possibility of choice between several proposals to perform experiments \cite{Kitchenham2002-GuideLinesESE}, results in uncertainties regarding the correct option to take. The lack or misuse of experimental design \cite{Dyba2006} increases the threats to validity around the experiments. The lack of communication between researchers \cite{Shull-2004-Knowledge-Sharing-Issues} generates problems around  sharing and storage of experimental information. The amount of experimental replication in SE is limited, mainly due to its complexity \cite{da-silva-2012-replication-on-empirical-studies}; hence, it's difficult to generalize the experimental results. The dearth of uniformity in the experimental reports \cite{Jedlitschka2008} makes difficult the information access. Among other problems.

Such mistakes in carrying out experimentation in SE \cite{Wainer-2009-Empirical-Evaluation} aren't privative of this discipline. Several other experimental disciplines, such as Medicine and Psychology, reported similar problems \cite{Garcia-2012-Cienci-Abierta}. Experimentation in such disciplines is currently a very streaked and uniform process \cite{Jones-2007-FUGE,ko-2012-SMISB}.

Esto nos ha motivado a indagar de forma detallada el modo en que se realizan y reportan experimentos en la SE, para luego compararlo con un estudio similar de una disciplina experimental asentada

In this research, we aim to study the underlying problems surrounding the ESE and inquire about their possible origins. To do this, we will carry out an empirical study based on a structured survey, in order to obtain insights of the most representative researchers into the Empirical Software Engineering Community (ESEC). From this point of view, we have considered survey the researchers whose articles have been accepted in the most representative symposiums, workshops or conferences of the ESEC, such as: the Symposium on Empirical Software Engineering and Measurement (ESEM), the Experimental Software Engineering Latin America Workshop (ESELAW), among others.



El presente trabajo consisti� en un estudio emp�rico exploratorio sobre el proceso de experimentaci�n realizado en una disciplina experimental consolidada, como es el caso de la Biotecnolog�a. M�s espec�ficamente, se realiz� un estudio etnogr�fico sobre el proceso experimental que se sigue en el laboratorio de ... de la carrera de Ingenier�a en Biotecnolog�a de la Universidad de las Fuerzas Armadas ESPE de Ecuador.

Como resultado de este estudio se obtuvieron modelos del proceso experimental de Biotecnolog�a, que permitieron hacer una comparaci�n con el proceso experimental de la SE. La comparaci�n entre los dos procesos evidenci� que las actividades llevadas a cabo en el proceso experimental de la Biotecnolog�a son similares a las de la ESE en su gran mayor�a. Sin embargo, existen actividades de la experimentaci�n en Biotecnolog�a que podr�an ser a�adidas dentro del proceso de la ESE para mejorarlo, particularmente en lo que respecta a la formalidad de la distribuci�n de actividades entre los roles que se tornan en el contexto experimental de esta disciplina.

El documento est� estructurado de la siguiente forma: En la Secci�n \ref{sec-background} se detallan los antecedentes en torno a la problem�tica actual de la ESE. El protocolo metodol�gico aplicado para la investigaci�n es detallado en la Secci�n III ...

NOTAS AUDIO

