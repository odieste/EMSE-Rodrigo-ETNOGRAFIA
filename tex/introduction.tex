\section{Introduction}\label{sec-introduction}
Experimentation is a well-established research method in Software Engineering (SE) nowadays. It is difficult to find conferences or journals in SE where experimental studies are not published. The fact of having some kind of empirical validation is today a practically unavoidable condition for a research can be published \cite{Weyuker-2011-ESE-Good-Bad-Ugly}.

SE researchers have started to perform an introspection on how empirical SE functions, as it had happened in other traditional disciplines, such as Medicine \cite{Baumgardner-1997-statistical-issues-medicine}. For example, Kitchenham et al. \cite{Kitchenham2002-GuideLinesESE} point it out that: \textquotedblleft If researchers have difficulty in a discipline such as medicine, which has a rich history of empirical research, it is hardly surprising that software engineering researchers have problems\textquotedblright. For instance, as an outcome of this exercise have already been identified: inadequate procedures of statistical analyses \cite{Reyes-2018-Statistical-Errors-in-SE}, experiment designs do not properly performed \cite{Vegas-2016-Crossover-Designs-ESE} and problems derived from researcher and publication bias \cite{Jorgensen-2016-Incorrects-Results-SEE}.

An issue that begins to attract more attention nowadays is the high cost of carrying out SE experiments in terms of difficulty, number of participants and effort. Some circumstances apparently depict this fact, for instance: abundance of experiments \cite{Sjoberg2005-surveyexperimentsESE,Zendler2001,Dieste2011-CompAnalMetaWhenWich}, with little replications \cite{da-silva-2012-replication-on-empirical-studies}; uncertainties at the time to choose the best way to perform \cite{Juristo2001,Wohlin2000,Wohlin2012-Experimentation,Basili-1986-ESE,Kitchenham2002-GuideLinesESE} and report \cite{Jedlitschka2008,Kitchenham:2008b,Carver2010-GuidelinesReplication} experiments; uncertainties at the time of selecting the adequate statistical technique to analyze experiments \cite{Reyes-2018-Statistical-Errors-in-SE}; among others.

This issue is not privative of SE. Other experimental disciplines, such as Biotechnology \cite{Kumar-2014-design-experiment-bioprocessing}, Medicine and Psychology, reported similar problems at some point \cite{Garcia-2012-Cienci-Abierta}. Thanks to such awareness, experimentation in those disciplines is currently a very rigorous and uniform process \cite{stoker-2009-iLAP-bioinformatics,Jones-2007-FUGE,ko-2012-SMISB}, a fact which shows that the problems have been diagnosed at the beginning, to be solved later. A similar situation is also apparently happening in SE, but possibly in a less formal way.


We performed a mix-methods empirical study to inquire the current state of the experimentation process' formalization into SE, compared with the traditional experimental disciplines. Firstly, we carried out a comparative observational study about the experimentation process that takes place in SE and a traditional experimental discipline, such as Biotechnology. Several representative research groups of both communities were studied to give validity to this study. However, since findings concerning a narrow range of samples are not significant, we surveyed the most representative researchers of the ESE community just to validate and generalize the findings obtained in the first study.

The results o this study contribute to SE research through the identification of experiment activities in another traditional discipline, which could enhance the formalization of experiment process in specific research areas of SE. We believe that such formalization is possible if the harmony of experiment tasks are achieved through particular protocols per research area. \rodrinote{\textquotedblleft COMPLETAR\textquotedblright}.

The remainder of the paper is organized as follows: Section \ref{sec-background} introduces the context which motivated the research. Research methodology is described in Section \ref{sec-methology}. Section \ref{sec-ESE-etnography} shows the results of the ethnographic study carried out in a representative SE research group. Afterward, Section \ref{sec-survey} details the results of the validation about SE experimentation issues identified. Section \ref{sec-bio-etnography} specifies the ethnographic study performed in specific Biotechnology areas. A summarized discussion about threats to validity identified is presented in Section \ref{sec-threats}. Finally, the discussion and main conclusions are presented in Section \ref{sec-discussion-conclusions}.

%More than 50 years have passed since the first experiment was carried out into SE \cite[p.~7]{Wohlin2012-Experimentation} as mechanism to assess current methods and techniques of software development in order to improve the software product quality. Several experiments have been performed both in academy \cite{Sjoberg2005-surveyexperimentsESE} and industry \cite{dieste-2013-industry-experiments}. Nowadays it is difficult to find conferences or journals that does not include experimental studies. This situation prompted that SE researchers, interested in experimentation, had to acquire adequate training to take advantage of the different sources of information regarding the experimental process (e.g.: booklets, textbooks, guidelines, papers, among others.)\cite[p.~3-4]{Kitchenham-2015-Literature-Review-Book}. Therefore, the experimentation in SE appears to be a seated methodology.

%Experimentation in SE appear to be a probed methodology, taking into account the amount of years that have passed since the first experiment was carried out and the vast literature generated in this discipline \cite[p.~7]{Wohlin2012-Experimentation}. Nowadays it is difficult to find conferences or journals in which are not included experimental studies.

%However, there remains important details within experimentation that apparently have not yet been fully suitable in SE and still represent a challenge.

%despite it has already been clearly identified in the past \cite{mian-200-computarized-infraestructure}

%There are some incidents within the experimental environment of SE that apparently depict this fact, as shown below.

%result in low trustworthiness of results reported in software engineering experiments.
%
%The lack or misuse of experimental design \cite{Dyba2006} increases the threats to validity around the experiments.
%
%The lack of communication between researchers \cite{Shull-2004-Knowledge-Sharing-Issues} generates problems around  sharing and storage of experimental information.
%
%The amount of experimental replication in SE is limited, mainly due to its complexity \cite{da-silva-2012-replication-on-empirical-studies}; hence, it's difficult to generalize the experimental results.
%
%The dearth of uniformity in the experimental reports \cite{Jedlitschka2008} makes difficult the information access.

% However, the current situation of the experimentation in SE is less satisfactory than it could be. Kitchenham et al. \cite{Kitchenham2002-GuideLinesESE}, for example, point it out that: \textquotedblleft If researchers have difficulty in a discipline such as medicine, which has a rich history of empirical research, it is hardly surprising that software engineering researchers have problems\textquotedblright.  

%However, in addition to this characteristics, they include the use of specialized vocabulary, and the decision making conflicting (on many occasions) with similar empirical studies, during the stages of design implementation and analysis, which causes that each process is completely different.

%Nowadays, there are uncertainties about how to perform experiments in terms of design \cite{Jorgensen-2016-Incorrects-Results-SEE} and execution \cite{mian-200-computarized-infraestructure}. According some studies the experimentation in SE has several deficiencies, for example: the possibility to follow different protocols to perform experiments, results in uncertainties regarding the correct option to take \cite{Wohlin2012-Experimentation}. The lack or misuse of experimental design \cite{Dyba2006} increases the threats to validity around the experiments. The lack of communication between researchers \cite{Shull-2004-Knowledge-Sharing-Issues} generates problems around  sharing and storage of experimental information. The amount of experimental replication in SE is limited, mainly due to its complexity \cite{da-silva-2012-replication-on-empirical-studies}; hence, it's difficult to generalize the experimental results. The dearth of uniformity in the experimental reports \cite{Jedlitschka2008} makes difficult the information access.

%Nowadays, there are uncertainties about how to perform experiments in terms of design \cite{Jorgensen-2016-Incorrects-Results-SEE} and execution \cite{mian-200-computarized-infraestructure}. According some studies the experimentation in SE has several methodological deficiencies, for example: the possibility to follow different protocols \cite{Juristo2001,Wohlin2000,Wohlin2012-Experimentation,Basili1985-ExperimentationSE,Kitchenham2002-GuideLinesESE,Jedlitschka2005-GuideLinesESE} to perform experiments results in uncertainties regarding the correct option to take.  Researcher and publication bias  \cite{Jorgensen-2016-Incorrects-Results-SEE} and inappropriate statistical procedures \cite{Reyes-2009-Statistical-Errors-in-SE}, such as the inappropriate use of techniques to formulate and evaluate statistical hypothesis, taking into account the meaningful conclusions obtained from it \cite{Miller-2004-statistical-significance};  

%Low trustworthiness of results reported in software engineering experiments mainly caused by researcher and publication bias, given typical statistical power and significance levels \cite{Jorgensen-2016-Incorrects-Results-SEE}.

%[OJO QUE ESTO AUMENT� AL PRINCIPIO, PERO DEBE QUEDAR COMO ANTES. ESTO PODR�A SER UTILIZADO M�S ABAJO EN DONDE HABLA DE LOS PROBLEMAS PUNTUALES: Nowadays, there are uncertainties about how to perform experiments in terms of avoid complex design with many statistical and unjustified tests \cite{Jorgensen-2016-Incorrects-Results-SEE} and execution \cite{Mendonca-2008-FIRE}.]

%Esto se corrobora al observar de cerca la realizaci�n de experimentos en dichas �reas, donde parece que el conocimiento es mucho m�s compartido. Esta caracter�stica induce a pensar en que la definici�n del protocolo experimental en otras disciplinas es m�s formal.

%Cabe preguntarse entonces \textquestiondown En qu� medida la diversidad de protocolos experimentales aplicados en la realizaci�n de un experimento desfavorece a la adopci�n de la experimentaci�n en SE? y \textquestiondown Cu�n madura es la experimentaci�n en SE respecto a otras disciplinas experimentales tradicionales?. These questions motivated to carry out an exploratory study to research about the underlying incidents surrounding the ESE and inquire about their possible origins. This empirical study was based on mix methods, in order to identificar los problemas en un entorno espec�fico donde se lleva cabo experimentaci�n en SE, obtain insights (survey) of the most representative researchers into the Empirical Software Engineering Community about ESE incidents y finalmente, comparar en la pr�ctica el proceso de ESE estudiado con el de una disciplina experimental tradicional.  

%In order to perform the investigation to give an answer to the RQ, a mixed research method was applied. First, we conducted an ethnographical study in a representative SE research group, in order to observe specific details about its experimental protocol. Secondly, we aimed at extrapolating our findings to the entire SE community by means of a survey. Finally, we conducted a second ethnographical study in a non-SE (biotechnology) research group, in order to compare our findings

%Como resultado de la investigaci�n encontramos que dentro de un grupo de investigaci�n en SE no se sigue un protocolo experimental �nico, dado que cada experimentador aplica su propia terminolog�a y criterio al realizar las actividades del proceso experimental que tiene a su cargo, lo que dificulta la comunicaci�n de informaci�n dentro y fuera del grupo, e influye negativamente en la integraci�n de informaci�n y en la realizaci�n de replicaciones. La comunidad de ESE al parecer esta consciente de esta situaci�n, lo que indica que esto nos es privativo del grupo estudiado, sino de toda la comunidad. Sin embargo, a pesar de que en las disciplinas experimentales maduras (como la Biotecnolog�a) existen los mismos problemas que en SE a nivel general, la diferencia se marca con la existencia de un protocolo experimental bien definido por �rea espec�fica (e.g.: nutrici�n, pat�genos de plantas, entre otras) que gu�a el proceso de experimentaci�n; lo cual, promueve la uniformidad del trabajo y la fluidez en la comunicaci�n entre los investigadores; dado que, cada experimentador realiza tareas puntuales en el proceso de experimentaci�n.