\section{Introduction}\label{sec-introduction}
SE researchers have started to perform introspection on how empirical Software Engineering (SE) functions, as it had happened in other traditional disciplines, such as Medicine \cite{altman-1998-statistical-reviewing-medical}. As Kitchenham et al. \cite{Kitchenham2002-GuideLinesESE} point it out: \textquotedblleft If researchers have difficulty in a discipline such as medicine, which has a rich history of empirical research, it is hardly surprising that software engineering researchers have problems\textquotedblright.

Los pocos estudios existentes han abordado principalmente aspectos estad�sticos. Por ejemplo, Vegas et al. \cite{Vegas-2016-Crossover-Designs-ESE} han estudiado en qu� medida los dise�os cross-over son correctamente analizados. Del mismo modo, \rodrinote{[COMPLETAR DE ACUERDO AL PAPER DE ROLANDO]} Desde una perspectiva m�s general, Reyes et al. \cite{Reyes-2018-Statistical-Errors-in-SE} han determinado emp�ricamente la existencia de distintos tipos de errores en experimentos de SE. J{\o}rgensen et al. \cite{Jorgensen-2016-Incorrects-Results-SEE} llegan a afirmar que una gran parte de los resultados experimentales en SE son falsos.

 This issue is not privative of SE. Other experimental disciplines, such as Biotechnology \cite{Kumar-2014-design-experiment-bioprocessing}, Medicine and Psychology, reported similar problems at some point \cite{Garcia-2012-Cienci-Abierta}. \rodrinote{[ESTE P�RRAFO DEBEB SER CAMBIADO DE ACUERDO AL PAPER DE ROLANDO]}
 
 Sin embargo, los aspectos matem�ticos

We performed a mix-methods empirical study to inquire the current state of the experimentation process' formalization into SE, compared with the traditional experimental disciplines. Firstly, we carried out a comparative observational study about the experimentation process that takes place in SE and a traditional experimental discipline, such as Biotechnology. Some representative research groups of both communities were studied to give validity to this study. However, we surveyed the most representative researchers of the ESE community in order to validate and generalize the findings obtained in the first study.

The results o this study contribute to SE research through the identification of experiment activities in another traditional discipline, which could enhance the formalization of experiment process in specific research areas of SE. We believe that such formalization is possible if the harmony of experiment tasks are achieved through particular protocols per research area. \rodrinote{\textquotedblleft COMPLETAR\textquotedblright}.

The remainder of the paper is organized as follows: Section \ref{sec-background} introduces the context which motivated the research. Research methodology is described in Section \ref{sec-methology}. Section \ref{sec-ESE-etnography} shows the results of the ethnographic study carried out in a representative SE research group. Afterward, Section \ref{sec-survey} details the results of the validation about SE experimentation issues identified. Section \ref{sec-bio-etnography} specifies the ethnographic study performed in specific Biotechnology areas. A summarized discussion about threats to validity identified is presented in Section \ref{sec-threats}. Finally, the discussion and main conclusions are presented in Section \ref{sec-discussion-conclusions}.
\rodrinote{OSCAR POR FAVOR REVISA NUEVAMENTE LA INTRODUCCI�N}.