\section{Introduction}\label{sec-introduction}
Software Engineering (SE) researchers have started to perform introspection on how Empirical SE functions, as it happened in other traditional disciplines, such as Medicine \cite{Altman-1998-statistical} and Psychology \cite{Anderson-1971-psychology-experiment,kruglanski-1975-psychology-experiment}, to cite two examples. As Kitchenham et al. \cite{Kitchenham2002-GuideLinesESE} point it out: \textquotedblleft If researchers have difficulty in a discipline such as medicine, which has a rich history of empirical research, it is hardly surprising that software engineering researchers have problems\textquotedblright.

Los pocos estudios existentes en SE han abordado principalmente aspectos estad�sticos. Por ejemplo, Vegas et al. \cite{Vegas-2016-Crossover-Designs-ESE} han estudiado en qu� medida los dise�os cross-over son correctamente analizados. Del mismo modo, Kitchenham et al. \cite{Kitchenham-2019-problems-statistical-practice} han identificado diversos problemas en art�culos experimentales publicados en high quality journals, e.g., an�lisis incorrectos, post-hoc power analysis, or multiple testing. Desde una perspectiva m�s general, Reyes et al. \cite{Reyes-2018-Statistical-Errors-in-SE} han determinado emp�ricamente la existencia de distintos tipos de errores en experimentos de SE, e.g., hip�tesis estad�sticas ausentes, falta de aleatorizaci�n, o la no comprobaci�n de los requisitos de los tests estad�sticos utilizados. Siguiendo a Ioannidis \cite{Ioannidis-2005-findings-false}, J{\o}rgensen et al. \cite{Jorgensen-2016-Incorrects-Results-SEE} llegan a afirmar que una gran parte de los resultados experimentales en SE son falsos.

Esto es una prueba.

 This issue is not privative of SE. Other experimental disciplines, such as medicine \cite{Welch-1996-review}, psychology \cite{Bakker-2011-mis}, and psychiatry \cite{Ercan-2008-misusage}, reported similar problems at some point. Sin embargo, los aspectos matem�tico-estad�sticos solo representan parte de la complejidad de la investigaci�n experimental, y hasta cierto punto lo que m�s f�cilmente puede ser gestionado. To the best of our knowledge, no existe ning�n estudio en SE que aborde el proceso experimental en general.
 
Este art�culo tiene como objetivo \textit{averiguar el modo en que los investigadores de SE afrontan la investigaci�n experimental, y cu�les son los problemas y desaf�os a los que se enfrentan}. Lo m�s pr�ximo a esta investigaci�n es la abundante bibliograf�a acerca de replicaciones experimentales, donde tambi�n se se�alan varias carencias, como por ejemplo: dise�os y reportes experimentales inadecuados para una replicaci�n \cite{Miller-2005-replicating-SE-experiments}, diversidad terminol�gica en los experimentos que imposibilitan su replicaci�n \cite{Demagalhaes-2015-SMS-Replications}. Las diferencias entre dichos trabajos, y este estudio, son las siguiente:
 \begin{itemize}
 	\item Nuestra investigaci�n est� enfocada en el modo en que los experimentadores en SE conciben, planifican, ejercitan y documentan una investigaci�n experimental, y no l�ricamente en la replicaci�n experimental.
 	\item We performed a mix-method study. First, we carried out a comparative observational study between the experimentation processes in SE and a traditional experimental discipline (Biotechnology). Second, we surveyed the experimental software community in order to validate and generalize the findings obtained in the first study.
 \end{itemize}
 
This study makes several contributions to SE empirical. On the one hand, we have identified some weaknesses in the way that SE experimenters design and conduct experiments. These weaknesses become evident when we compare SE and the practices of other experimental disciplines. On the other, we provide several actionable measures to alleviate the observed weaknesses and progress towards a more mature experimental practice, in line with traditional experimental disciplines.

The remainder of the paper is organized as follows: Section \ref{sec-background} introduces the context which motivated the research. Research methodology is described in Section \ref{sec-methology}. Section \ref{sec-ESE-etnography} shows the results of the ethnographic study carried out in a representative SE research group. Afterward, Section \ref{sec-survey} details the results of the validation about SE experimentation issues identified. Section \ref{sec-bio-etnography} specifies the ethnographic study performed in specific Biotechnology areas. A summarized discussion about threats to validity identified is presented in Section \ref{sec-threats}. Finally, the discussion and main conclusions are presented in Section \ref{sec-discussion-conclusions}.