\section{Conclusions}\label{sec-discussion-conclusions}

El estudio etnogr\'afico realizado muestra que existen diferencias entre el proceso experimental, tal y como es concebido en la teor\'ia, y su aplicaci\'on pr\'actica. Las descripciones que los experimentadores hacemos de los experimentos deben mucho al modo en que los reportamos, pero el reporte es s\'olo una conceptualizaci\'on final del proceso investigativo, que muchas veces diverge del modo en que los experimentos fueron realmente dise\~nados y ejecutados \cite{medawar1990scientific}. Creemos que muchos experimentadores de la ESE se ver\'an reconocidos en las observaciones que hemos realizado, aunque los detalles concretos pueden variar dependiendo del grupo de investigaci\'on involucrado.

Las conclusiones del estudio arrojan luz sobre varios aspectos relevantes de la ESE. El primero, y m\'as sencillo, es el tema de la educaci\'on de los investigadores. El amateurismo provoca problemas de comunicaci\'on, lo que a su vez complica la replicaci\'on experimental.�No hemos sido los primeros en identificar este problema \odnote{RODRIGO: necesito citas}\odnote{Poner citas del paper de las entrevistas de Rolando} y est\'a claro que hay que solucionarlo. Sin embargo, creemos que la soluci\'on no pasa simplemente por establecer un curriculum; es necesario algo m\'as.

Dos cosas que hemos aprendido durante la investigaci\'on son que: 1) Es poco probable que se pueda llegar a un acuerdo dentro de la comunidad ESE acerca de un proceso experimental general, ya que 2) cada \'area de investigaci\'on tiene caracter\'isticas particulares. Por poner un ejemplo, aunque superficialmente un experimento de \textit{mutation testing} es similar a un experimento de \textit{requirements elicitation}, las claves para entender dichos experimentos podr\'ian est\'an en el modo de generaci\'n de los mutantes o en los modelos conceptuales del dominio del problema, respectivamente. Sin una formaci\'on espec\'ifica, los experimentadores en testing no van a entender mejor el experimento de elicitaci\'on, y viceversa, por mucho que etiquetemos dichos conceptos como ''objetos experimentales.'' Y lo que es peor; una vez comprendidos dichos conceptos, la etiqueta ''objeto experimental'' puede resultar excesivamente general, y por lo tanto de poco valor para el experimentador conocedor de su \'area.

Por este motivo, creemos que es mucho m�s importante abordar la educaci�n en experimentaci�n dentro �reas concretas, de modo que se alcance convergencia tanto a nivel terminol�gico como operativo. En otras palabras, abogamos por una especializaci\'on 


\item El trabajo dentro de �reas traer� como beneficio adicional la especializaci�n: uniformidad en protocolos y m�todos, pericias espec�ficas de los investigadores que desembocar�n en roles bien definidos


En ESE, hemos prestado mucha atenci\'on al an\'alisis y reporte de experimentos. 




\begin{itemize}


\item Es posible, aunque no seguro, que la especializaci�n mejore la replicaci�n. Pero eso no es seguro. El problema de la replicaci�n es bastante general y no es muy cre�ble que lo hayan solucionado en Biotecnolog�a. Quiz�s repliquen mejor que otros, pero dudamos que repliquen sin ning�n tipo de problema.
\item incidentalmente, mejorara las bases de datos experimentales

\item Como futuras l�neas: (1) trabajar en la estandarizaci�n del proceso y protocolos, tanto a nivel operativo y termon�logico. Es el primer paso indicado arriba, y probablemente el m�s importante de todos.
\end{itemize}

