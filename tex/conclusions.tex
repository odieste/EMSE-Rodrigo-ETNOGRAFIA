\section{Conclusions}\label{sec-discussion-conclusions}

\begin{itemize}
\item No vendr�a mal llegar a alg�n tipo de proceso experimental estadarizado
\item Sin embargo, es mucho m�s importante trabajar en la experimentaci�n dentro �reas concretas, de modo que se alcance convergencia tanto a nivel terminol�gico como operativo.
\item El trabajo dentro de �reas traer� como beneficio adicional la especializaci�n: uniformidad en protocolos y m�todos, pericias espec�ficas de los investigadores que desembocar�n en roles bien definidos
\item Es posible, aunque no seguro, que la especializaci�n mejore la replicaci�n. Pero eso no es seguro. El problema de la replicaci�n es bastante general y no es muy cre�ble que lo hayan solucionado en Biotecnolog�a. Quiz�s repliquen mejor que otros, pero dudamos que repliquen sin ning�n tipo de problema.
\item Educaci�n: necesitamos m�s, pero quiz�s no sea la clave.
\item Como futuras l�neas: (1) trabajar en la estandarizaci�n del proceso y protocolos, tanto a nivel operativo y termon�logico. Es el primer paso indicado arriba, y probablemente el m�s importante de todos.
\end{itemize}

