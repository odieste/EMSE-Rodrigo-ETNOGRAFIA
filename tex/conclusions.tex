\section{Conclusions}\label{sec-discussion-conclusions}

El estudio etnogr\'afico realizado muestra que existen diferencias entre el proceso experimental, tal y como es concebido en la teor\'ia, y su aplicaci\'on pr\'actica. Las descripciones que los experimentadores hacemos de los experimentos deben mucho al modo en que los reportamos, pero el reporte es s\'olo una conceptualizaci\'on final del proceso investigativo, que muchas veces diverge del modo en que los experimentos fueron realmente dise\~nados y ejecutados \cite{medawar1990scientific}. Creemos que muchos experimentadores de la ESE se ver\'an reconocidos en las observaciones que hemos realizado, aunque los detalles concretos pueden variar dependiendo del grupo de investigaci\'on involucrado.

Las conclusiones del estudio arrojan luz sobre varios aspectos relevantes de la ESE. El primero, y m\'as sencillo, es el tema de la educaci\'on de los investigadores. El amateurismo provoca problemas de comunicaci\'on, lo que a su vez complica la replicaci\'on experimental.�No hemos sido los primeros en identificar este problema \odnote{RODRIGO: necesito citas}\odnote{Poner citas del paper de las entrevistas de Rolando} y est\'a claro que hay que solucionarlo. Sin embargo, creemos que la soluci\'on no pasa simplemente por establecer un curriculum; es necesario algo m\'as.

Dos cosas que hemos aprendido durante la investigaci\'on son que: 1) Es poco probable que se pueda llegar a un acuerdo dentro de la comunidad ESE acerca de un proceso experimental general, ya que 2) cada \'area de investigaci\'on tiene caracter\'isticas particulares. Por poner un ejemplo, aunque superficialmente un experimento de \textit{mutation testing} es similar a un experimento de \textit{requirements elicitation}, las claves para entender dichos experimentos podr\'ian est\'an en el modo de generaci\'n de los mutantes o en los modelos conceptuales del dominio del problema, respectivamente. Sin formaci\'on espec\'ifica, los experimentadores en testing no van a entender mejor el experimento de elicitaci\'on, y viceversa, por mucho que etiquetemos dichos conceptos como ''objetos experimentales.'' Y lo que es peor; una vez comprendidos dichos conceptos, la etiqueta ''objeto experimental'' puede resultar excesivamente general, y por lo tanto de poco valor para el experimentador conocedor del \'area de investigaci\'on.

Por este motivo, creemos que es mucho m�s importante abordar la educaci\'on en experimentaci�n dentro �reas concretas, no a nivel general. Cada \'area deber\'ia desarrollar su terminolog\'ia y procedimientos propios. En cierto sentido, esto ya est\'a ocurriendo. A principios de los a\~nos 2000, eran pocos los art\'iculos que inclu\'ian una validaci\'on experimental y, por ello, dichos art\'iculos se publicaban en foros especializados como el ISESE (\textit{International Symposium on Empirical Software Engineering}). Actualmente, los art\'iculos con validaci\'on experimental se publican en cualquier revista o conferencia. Ser\'ia necesario dar un \'ultimo paso y crear libros de texto especializados por \'area que permitieran alcanzar la convergencia tanto a nivel terminol�gico como operativo.

Conviene aclarar que cuando hablamos de libros de texto, no nos referimos a procedimientos de an\'alisis o est\'andares de reporte. En nuestra investigaci\'on, el an\'alisis y el reporte no han resultado problem\'aticos de realizar en ESE, ni de entender en Biotecnolog�a. Nos referimos por lo tanto a las actividades de planificaci\'on, dise\~no, ejecuci\'on y medici\'on, las cuales fueron m\'as complicadas de realizar durante la replicaci\'on experimental realiza en el marco de la observaci\'on participativa (ver Secci\'on~\ref{subsec-second-aprox}).

Por \'ultimo, es posible que la especializaci�n mejore la replicaci�n, pero no es seguro. El problema de la replicaci�n es bastante general y no es muy cre�ble que haya sido solucionado en Biotecnolog�a. Quiz�s repliquen mejor que en ESE, pero dudamos que repliquen sin ning�n tipo de problema. En caso contrario, el concepto de ''lab mythology'', desarrollado en las ciencias, no estar\'ia justificado.
