\section{Background}\label{sec-background}
Desde sus inicios la experimentaci�n en SE ha pretendido consolidarse a trav�s de propuestas metodol�gicas que gu�en su proceso \cite{Basili-1986-ESE,Juristo2001,Kitchenham2002-GuideLinesESE,Wohlin2000}, las replicaciones \cite{Solari2006-ClasAnalLabPaq} e incluso sus reportes \cite{Jedlitschka2005-GuideLinesESE,Carver2010-GuidelinesReplication,Kitchenham:2008b}; de hecho, las instancias de experimentos se han incrementado notablemente en los �ltimos tiempos \cite{Sjoberg2005-surveyexperimentsESE,Zendler2001,Dieste2011-CompAnalMetaWhenWich}, a su vez la importancia de sus hallazgos e indica que este paradigma est� bien posicionado en SE.

No obstante, en la pr�ctica se aplican una diversidad de protocolos y formatos de reporte que superan abismalmente las propuestas formales antes mencionadas. Esta condici�n da indicios para asumir que cada experimento en SE incluye actividades diversas que han sido adoptadas de acuerdo a la concepci�n particular de cada investigador o grupo de investigadores y causa incertidumbre en torno al efecto que esto podr�a implicar. En la literatura se han publicado diversos estudios que sugieren circunstancias que podr�an corroboran esta situaci�n, de los cuales a continuaci�n se citan algunos ejemplos.

Magalh�es et al. \cite{Demagalhaes-2015-SMS-Replications} indican que \textquotedblleft Replications of empirical studies play important roles in the construction of knowledge\textquotedblright. No obstante, en el mismo estudio se menciona que las replicaciones en SE son muy complejas de realizar, debido probablemente (entre otras causas) a las casi inevitables variaciones que existen entre el estudio original y la replicaci�n por razones intencionadas o no intencionadas. Esto podr�a justificar la carencia de armon�a entre el creciente n�mero de experimentos en SE y el n�mero de replicaciones existentes que indican da Silva et al. en su estudio \cite{da-silva-2012-replication-on-empirical-studies}. 


 

Por el contrario, otras disciplinas experimentales tradicionales tienen currently a very rigorous and uniform process \cite{stoker-2009-iLAP-bioinformatics,Jones-2007-FUGE,ko-2012-SMISB}. No obstante, los investigadores en SE han iniciando la realizaci�n de una introspecci�n en su disciplina encontrando de la SE es una disciplina candidata a que en sus instancias experimentales se encuentren errores