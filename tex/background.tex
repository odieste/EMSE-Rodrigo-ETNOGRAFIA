\section{Background}\label{sec-background}
Desde sus inicios la experimentaci�n en SE ha pretendido consolidarse principalmente a trav�s de propuestas metodol�gicas que gu�en su proceso \cite{Basili-1986-ESE,Juristo2001,Kitchenham2002-GuideLinesESE,Wohlin2000}, sus replicaciones \cite{Solari2006-ClasAnalLabPaq} e incluso sus reportes \cite{Jedlitschka2005-GuideLinesESE,Carver2010-GuidelinesReplication,Kitchenham:2008b}, lo que podr�a haber influido de cierta manera en el notable incremento de instancias de experimentos publicadas en los �ltimos tiempos \cite{Sjoberg2005-surveyexperimentsESE,Zendler2001,Dieste2011-CompAnalMetaWhenWich}. No obstante, las publicaciones de revistas y congresos representativos de la comunidad ESE muestran la utilizaci�n de una basta diversidad de protocolos y formatos de reporte que rebasan ampliamente las propuestas metodol�gicas antes mencionadas, y dan muestras de que cada experimento reportado podr�a haber incluido actividades concebidas de la percepci�n particular de cada investigador o grupo de investigadores.

Al parecer, there remains important details within experimentation that apparently have not yet been fully suitable in SE and still represent a challenge. En la literatura se han publicado diversos estudios que reportan circunstancias que podr�an responder a esta situaci�n, de los cuales se citan algunos ejemplos a continuaci�n.

Magalh�es et al. \cite{Demagalhaes-2015-SMS-Replications} state that \textquotedblleft Replications of empirical studies play important roles in the construction of knowledge\textquotedblright. Additionally, authors mentioned that SE replications are  complex tasks to perform, due probably (among other causes) to a las casi inevitables variaciones que existen entre el estudio original y la replicaci�n por razones intencionadas o no intencionadas. Por otra parte, Carver et al. \cite{Carver-2014-Replications-of-SE} de forma concordante indican que \textquotedblleft Replication is an essential part of the experimental paradigm and is considered the cornerstone of scientific knowledge \textquotedblright; as� mismo indican que \textquotedblleft Replication of empirical software engineering studies has not yet attracted enough attention from researchers\textquotedblright. Es dif�cil entender este comportamiento inconsistente con la indudable importancia de las replicaciones en la construcci�n del conocimiento. Si bien es cierto, es complicado llevar a cabo una replicaci�n en SE; no obstante, esto no justifica que los experimentos en SE no sean replicados \cite{da-silva-2012-replication-on-empirical-studies}. 

Por el contrario, otras disciplinas experimentales tradicionales tienen currently a very rigorous and uniform process \cite{stoker-2009-iLAP-bioinformatics,Jones-2007-FUGE,ko-2012-SMISB}.

There are groups of similar experimental studies (statistically speaking) applying different statistical techniques for their analysis, that in some cases incurre in methodological errors, which cause uncertainty at the time of opting for one of the several existing experiments' analysis techniques \cite{Reyes-2009-Statistical-Errors-in-SE}.

Practitioners still have serious doubts about the validity of findings obtained from empirical research performed in SE \cite{Lo-2015-Practitioners-Perceive}, despite the vast amount of studies carried out in academia \cite{Sjoberg2005-surveyexperimentsESE} and lately in industry \cite{dieste-2013-industry-experiments}. 

Como se puede notar es com�n la presencia de eventos eventos pe
Esta situaci�n nos ha motivado ha a realizar esta investigaci�n


%\begin{itemize}
%  \item There is a growing trend in SE to carry out experiments \cite{Sjoberg2005-surveyexperimentsESE,Zendler2001,Dieste2011-CompAnalMetaWhenWich}, but the percentage of replications performed so far does not harmonize to such experimentation \cite{da-silva-2012-replication-on-empirical-studies}, which could minimize the validity of its findings.
%  \item According to several authors \cite{Demagalhaes-2015-SMS-Replications,Gomez-2014-understanding-replication,Carver-2014-Replications-of-SE}, the lack of replications in SE is due primarily to there is uncertainty at the time of opting for one of the several existing mechanisms of replication, and that, none of them has been totally effective.
%  \item There are basic guidelines to perform experiments and hence replications in SE \cite{Juristo2001,Wohlin2000,Wohlin2012-Experimentation,Basili-1986-ESE,Kitchenham2002-GuideLinesESE}, as well as to report it \cite{Jedlitschka2008,Kitchenham:2008b,Carver2010-GuidelinesReplication}; however, the variety of experimental protocols and report formats applied in practice far exceeds those methodological proposals \cite{Demagalhaes-2015-SMS-Replications}.
%\end{itemize}
