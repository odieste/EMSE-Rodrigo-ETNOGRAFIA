\section{Background}\label{sec-background}
SE researchers have started to perform introspection on how the experimentation in SE functions. The scarce studies in this regard coincide with report unexpected or peculiar behaviors surrounding the experimentation process. This process is not privative of SE. Other experimental disciplines realized similar problems at some point \cite{Garcia-2012-Cienci-Abierta,Kumar-2014-design-experiment-bioprocessing}. Thanks to such awareness, experimentation in those disciplines is currently a very rigorous and uniform process \cite{stoker-2009-iLAP-bioinformatics,Jones-2007-FUGE,ko-2012-SMISB}. This fact could reveal that there apparently remain essential details within experimentation have not yet been entirely suitable in SE \cite{Vegas-2016-Crossover-Designs-ESE}. Below are some examples of facts that occur in the SE experimentation practice.

Experimentation in SE since its inception has sought to strengthen itself mainly through methodological proposals that guide its process \cite{Basili-1986-ESE,Juristo2001,Kitchenham2002-GuideLinesESE,Wohlin2000}, its replications \cite{Solari2006-ClasAnalLabPaq}, and even its reports \cite{Jedlitschka2005-GuideLinesESE,Carver2010-GuidelinesReplication,Kitchenham:2008b}. This strategy could have influenced in the important increase of experimental instances \cite{Sjoberg2005-surveyexperimentsESE,Zendler2001,Dieste2011-CompAnalMetaWhenWich} published so far. Nevertheless, the reality demonstrates totally the opposite, since publications of most important journals and conferences of the ESE community show the use of a wide variety of experimental protocols and reporting formats, different from those suggested in the aforementioned methodological proposals. This fact apparently shows each experiment reported could have included activities conceived as part of the particular perception of each researcher or group of researchers \cite{Reyes-2018-Statistical-Errors-in-SE}. Besides, the existence of a wide variety of options to do or to report an experiment could cause uncertainty at the time of selecting the most suitable option.

Practitioners still have severe doubts about the validity of findings obtained from empirical research performed in SE \cite{Lo-2015-Practitioners-Perceive}, despite the vast amount of experiments carried out in academia \cite{Sjoberg2005-surveyexperimentsESE} and lately in industry \cite{dieste-2013-industry-experiments}, possibly due to there are an insufficient amount of replications \cite{da-silva-2012-replication-on-empirical-studies} concerning to such amount of experiments \cite{Sjoberg2005-surveyexperimentsESE,dieste-2013-industry-experiments}. It is difficult to understand this unexpected behavior given the undoubted importance of replications in the construction of scientific knowledge \cite{Demagalhaes-2015-SMS-Replications,Carver-2014-Replications-of-SE}. There are authors who do not ignore the importance of replications in SE, but they warned \textquotedblleft Replication of empirical software engineering studies has not yet attracted enough attention from researchers\textquotedblright \cite{Carver-2014-Replications-of-SE}. Other authors mention that SE replications are complex tasks to perform, due probably (among other causes) to the almost unavoidable variations existing between the original study and its replications for intended or unintended reasons \cite{Shull2002-TacitKnoeledge,Gomez-Juristo-2014-Understanding-replication}.

Some essential activities under the scope of experimentation process are carried out through not specific methods and techniques,  which cause uncertainty at the time of choosing the most appropriate and even going to incur in methodological errors \cite{Vegas-2016-Crossover-Designs-ESE,Reyes-2018-Statistical-Errors-in-SE,Jorgensen-2016-Incorrects-Results-SEE}. For example, different statistical techniques are applied for the analysis of groups of similar experimental studies, which cause uncertainty when opting for one of the several existing experiments' analysis techniques \cite{Reyes-2018-Statistical-Errors-in-SE}. 

Researchers are slowly becoming aware (or accepting) of the use of questionable research practices in SE \cite{Jorgensen-2016-Incorrects-Results-SEE}. Hence, they are trying to improve some activities within the experimentation process. However, this awareness too passive is extremely worrying, which motivate us to performed an exploratory comparative study directly in the trenches where experimentation is cooked, in order to know the causes by which experimentation in SE does not take off. Our study was made both in the SE and in a traditional discipline.
%SE researchers have started to perform introspection on how the ESE functions. The first outcome of such analysis is there apparently remain essential details within experimentation have not yet been entirely suitable in SE, which still represent a challenge and could be a cause to increase the cost of carrying out SE experiments \cite{Reyes-2018-Statistical-Errors-in-SE,Vegas-2016-Crossover-Designs-ESE}. However, there are few studies published in the literature reporting unexpected or peculiar actions, which could confirm this situation. Below are some examples of facts that occur in the SE experimentation practice.

%Replications represents a fundamental piece in the construction of knowledge according to several authors. For example, Magalh�es et al. \cite{Demagalhaes-2015-SMS-Replications} state that \textquotedblleft Replications of empirical studies play essential roles in the construction of knowledge\textquotedblright. On the other hand, Carver et al. \cite{Carver-2014-Replications-of-SE} state that \textquotedblleft Replication is an essential part of the experimental paradigm and is considered the cornerstone of scientific knowledge \textquotedblright. 

%%No obstante, existen otros autores que, aunque no desconocen la importancia de las replicaciones, indican que particularmente en SE las replicaciones representan un reto. Por ejemplo,   Carver et al. \cite{Carver-2014-Replications-of-SE} state that \textquotedblleft Replication of empirical software engineering studies has not yet attracted enough attention from researchers\textquotedblright. As� mismo, other authors like \cite{Shull2002-TacitKnoeledge,Gomez-Juristo-2014-Understanding-replication} mention that SE replications are complex tasks to perform, due probably (among other causes) to the almost unavoidable variations existing between the original study and its replications for intended or unintended reasons. Esto podr�a 


%\begin{itemize}
%  \item There is a growing trend in SE to carry out experiments \cite{Sjoberg2005-surveyexperimentsESE,Zendler2001,Dieste2011-CompAnalMetaWhenWich}, but the percentage of replications performed so far does not harmonize to such experimentation \cite{da-silva-2012-replication-on-empirical-studies}, which could minimize the validity of its findings.
%  \item According to several authors \cite{Demagalhaes-2015-SMS-Replications,Gomez-2014-understanding-replication,Carver-2014-Replications-of-SE}, the lack of replications in SE is due primarily to there is uncertainty at the time of opting for one of the several existing mechanisms of replication, and that, none of them has been totally effective.
%  \item There are basic guidelines to perform experiments and hence replications in SE \cite{Juristo2001,Wohlin2000,Wohlin2012-Experimentation,Basili-1986-ESE,Kitchenham2002-GuideLinesESE}, as well as to report it \cite{Jedlitschka2008,Kitchenham:2008b,Carver2010-GuidelinesReplication}; however, the variety of experimental protocols and report formats applied in practice far exceeds those methodological proposals \cite{Demagalhaes-2015-SMS-Replications}.
%\end{itemize}
