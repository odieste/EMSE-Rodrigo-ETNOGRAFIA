\section{Background}\label{sec-background}
Desde sus inicios la experimentaci�n en ingenier�a de software ha pretendido consolidarse a trav�s de propuestas metodol�gicas que gu�en el proceso experimental \cite{Basili-1986-ESE,Juristo2001,Kitchenham2002-GuideLinesESE,Wohlin2000}, las replicaciones \cite{Solari2006-ClasAnalLabPaq} e incluso los reportes experimentales \cite{Jedlitschka2005-GuideLinesESE,Carver2010-GuidelinesReplication,Kitchenham:2008b}. De hecho, la experimentaci�n en SE se ha incrementado notablemente en los �ltimos tiempos y sus hallazgos son alentadores, lo que indica que es un paradigma que est� bien posicionado.

No obstante, en la pr�ctica existe una diversidad de protocolos experimentales aplicados y formatos de reporte utilizados que supera abismalmente las propuestas mencionadas. Esta condici�n da indicios para asumir que cada investigador en SE realiza experimentos de acuerdo a su propia concepci�n, lo cual causa incertidumbre en torno al efecto que esto podr�a causar. De hecho, existen diversas circunstancias que corroboran esta condici�n.

 

Por el contrario, otras disciplinas experimentales tradicionales tienen currently a very rigorous and uniform process \cite{stoker-2009-iLAP-bioinformatics,Jones-2007-FUGE,ko-2012-SMISB}.