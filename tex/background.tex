\section{Background}\label{sec-background}
Desde los inicios de la experimentaci�n en ingenier�a de software se ha intentado consolidar el proceso, posiblemente inspirados en la aparente formalidad de las disciplinas experimentales tradicionales. Tempranamente han aparecido propuestas para guiar el proceso experimental \cite{Basili-1986-ESE,Juristo2001,Kitchenham2002-GuideLinesESE,Wohlin2000}, las replicaciones \cite{Solari2006-ClasAnalLabPaq} e incluso los reportes experimentales \cite{Jedlitschka2005-GuideLinesESE,Carver2010-GuidelinesReplication,Kitchenham:2008b}. Sin embargo, la diversidad de protocolos experimentales utilizados y de formatos utilizados para su reporte, supera abismalmente las propuestas formales presentadas, lo que da muestra que cada investigador lleva a cabo la experimentaci�n de acuerdo a su concepci�n particular.   

[OJO: AQU� NO SE SI HABLAR DIRECTAMENTE DE LA EXPERIMENTACI�N O INICIAR DESDE UN PLANO M�S GENERAL, PARA ABORDAR OTROS M�TODOS EMP�RICOS QUE TAMBIEN TIENEN INCIDENTES CLAROS DE APLICACI�N DE DIVERSIDAD DE PROCESOS, COMO ES EL CASO POR EJEMPLO DE UNA SLR O SMS.]

Llama mucho esta situaci�n, ya que posiblemente la forma de llevar a cabo la experimentaci�n tenga que ver con 

No obstante, en la actualidad al parecer el proceso de experimentaci�n en SE se est� se est� reorientando en �reas espec�ficas, en el hecho de que los experimentadores tienden a realizar su labor de forma homog�nea siguiendo el protocolo de otros experimentadores \cite{Reyes-2009-Statistical-Errors-in-SE}. Sin embargo, existen diversas circunstancias que muestran que a�n queda mucho por hacer para formalizar el protocolo experimental.



