\section{Background}\label{sec-background}
Desde sus inicios la experimentaci�n en SE ha pretendido consolidarse a trav�s de propuestas metodol�gicas que gu�en su proceso \cite{Basili-1986-ESE,Juristo2001,Kitchenham2002-GuideLinesESE,Wohlin2000}, replicaciones \cite{Solari2006-ClasAnalLabPaq} e incluso reportes \cite{Jedlitschka2005-GuideLinesESE,Carver2010-GuidelinesReplication,Kitchenham:2008b}, lo que al parecer ha influido de cierta manera en que las instancias de experimentos se hayan incrementado notablemente en los �ltimos tiempos \cite{Sjoberg2005-surveyexperimentsESE,Zendler2001,Dieste2011-CompAnalMetaWhenWich}. However, there remains important details within experimentation that apparently have not yet been fully suitable in SE and still represent a challenge. En la literatura se han publicado diversos estudios que sugieren circunstancias que podr�an responder a esta situaci�n, de los cuales a continuaci�n se citan algunos ejemplos.

Magalh�es et al. \cite{Demagalhaes-2015-SMS-Replications} indican que \textquotedblleft Replications of empirical studies play important roles in the construction of knowledge\textquotedblright. No obstante, en el mismo estudio se menciona que las replicaciones en SE son muy complejas de realizar, debido probablemente (entre otras causas) a las casi inevitables variaciones que existen entre el estudio original y la replicaci�n por razones intencionadas o no intencionadas. Por otra parte, Carver et al. \cite{Carver-2014-Replications-of-SE} de forma concordante indican que \textquotedblleft Replication is an essential part of the experimental paradigm and is considered the cornerstone of scientific knowledge \textquotedblright, pero tambi�n indican que \textquotedblleft Replication of empirical software engineering studies has not yet attracted enough attention from researchers\textquotedblright; lo cual devela que no es

Esto podr�a justificar la carencia de armon�a entre el creciente n�mero de experimentos en SE y el n�mero de replicaciones existentes que indican da Silva et al. en su estudio \cite{da-silva-2012-replication-on-empirical-studies}.

A pesar de la existencia de las propuestas metodol�gicas antes mencionadas, en la pr�ctica se aplican una diversidad de protocolos y formatos de reporte que las superan abismalmente. Esta condici�n da indicios para asumir que cada experimento en SE incluye actividades diversas que han sido adoptadas de acuerdo a la concepci�n particular de cada investigador o grupo de investigadores y causa incertidumbre en torno al efecto que esto podr�a implicar. 



 

Por el contrario, otras disciplinas experimentales tradicionales tienen currently a very rigorous and uniform process \cite{stoker-2009-iLAP-bioinformatics,Jones-2007-FUGE,ko-2012-SMISB}. No obstante, los investigadores en SE han iniciando la realizaci�n de una introspecci�n en su disciplina encontrando de la SE es una disciplina candidata a que en sus instancias experimentales se encuentren errores


\begin{itemize}
  \item There is a growing trend in SE to carry out experiments \cite{Sjoberg2005-surveyexperimentsESE,Zendler2001,Dieste2011-CompAnalMetaWhenWich}, but the percentage of replications performed so far does not harmonize to such experimentation \cite{da-silva-2012-replication-on-empirical-studies}, which could minimize the validity of its findings.
  \item According to several authors \cite{Demagalhaes-2015-SMS-Replications,Gomez-2014-understanding-replication,Carver-2014-Replications-of-SE}, the lack of replications in SE is due primarily to there is uncertainty at the time of opting for one of the several existing mechanisms of replication, and that, none of them has been totally effective.
  \item There are basic guidelines to perform experiments and hence replications in SE \cite{Juristo2001,Wohlin2000,Wohlin2012-Experimentation,Basili-1986-ESE,Kitchenham2002-GuideLinesESE}, as well as to report it \cite{Jedlitschka2008,Kitchenham:2008b,Carver2010-GuidelinesReplication}; however, the variety of experimental protocols and report formats applied in practice far exceeds those methodological proposals \cite{Demagalhaes-2015-SMS-Replications}.
  \item There are groups of similar experimental studies (statistically speaking) applying different statistical techniques for their analysis, that in some cases incurre in methodological errors, which cause uncertainty at the time of opting for one of the several existing experiments' analysis techniques \cite{Reyes-2009-Statistical-Errors-in-SE}.
  \item Practitioners still have serious doubts about the validity of findings obtained from empirical research performed in SE \cite{Lo-2015-Practitioners-Perceive}, despite the vast amount of studies carried out in academia \cite{Sjoberg2005-surveyexperimentsESE} and lately in industry \cite{dieste-2013-industry-experiments}. 
\end{itemize}
