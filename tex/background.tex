\section{Background}\label{sec-background}
Desde sus inicios, la experimentaci�n en ingenier�a de software ha intentado consolidar su proceso a trav�s de propuestas metodol�gicas que gu�en el proceso experimental \cite{Basili-1986-ESE,Juristo2001,Kitchenham2002-GuideLinesESE,Wohlin2000}, las replicaciones \cite{Solari2006-ClasAnalLabPaq} e incluso los reportes experimentales \cite{Jedlitschka2005-GuideLinesESE,Carver2010-GuidelinesReplication,Kitchenham:2008b}. No obstante, la diversidad de protocolos experimentales aplicados y de formatos para reporte utilizados, supera ampliamente tales propuestas, lo que podr�a indicar que cada investigador lleva a cabo la experimentaci�n de acuerdo a su concepci�n particular. 

De la mano de esta pr�ctica, la experimentaci�n en SE se ha incrementado aceleradamente en los �ltimos tiempos. Los hallazgos obtenidos hasta ahora en la ESE son alentadores, lo que indica que es un paradigma que est� bien posicionado. Sin embargo, existen diversas circunstancias que muestran que a�n queda por hacer respecto a la formalizaci�n del protocolo experimental. Los primeros pasos se est�n dando, de hecho los investigadores en SE est�n haciendo una introspecci�n dentro la misma disciplina.



